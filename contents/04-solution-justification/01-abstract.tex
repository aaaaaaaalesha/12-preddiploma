\section{Обоснование решения математической постановки задачи}

В конечном итоге, обнаружение аномалий в сетевом трафике решено построить на базе автокодировщиков. Этот выбор обусловлен успешностью применения данного класса моделей для решения поставленной задачи и оптимизации параметра $\beta$ при установленном значении порога $\alpha$, что показано в результатах работ \cite{Gurina2019AnomalyBasedMF}, \cite{Applied-Autoencoder-One}, \cite{Applied-Autoencoder-Two}. Также это подтвердили численные эксперименты, проводимые в рамках прошедших НИРС.

Автокодировщики обладают способностью эффективно обрабатывать высокоразмерные данные, что является важным аспектом при анализе сетевого трафика, где количество признаков может быть значительным. Они способны выявлять сложные паттерны в данных благодаря своей архитектуре. Это позволяет им адаптироваться к различным типам данных и выявлять аномалии, которые могут быть неочевидны для других методов.

Также данный метод позволяет осуществить тонкую настройку за счёт установки параметров скрытых полносвязных слоёв нейросети автоэнкодера, выбора оптимальных функций активации, применения метода регуляризации \textit{Dropout} с подбором процента исключения случайных нейронов, что повышает итоговое качество предсказаний модели, а также подбор функции потерь и оптимизатора нейронной сети. 