\introduction

В современном информационном обществе огромное количество данных поступает из различных источников, таких как сети передачи данных, датчики, мобильные устройства и облачные платформы. Обработка и анализ этих данных стали неотъемлемой частью многих сфер, однако одним из серьезных вызовов является обнаружение аномалий в данных, которые могут свидетельствовать о наличии угроз безопасности или нештатных ситуациях. В этом контексте, проблема обнаружения аномалий в данных выходит на первый план. Аномалии могут служить индикаторами возможных атак, ошибок в данных или других проблем, которые могут существенно повлиять на надежность и безопасность информационных систем.

Под аномалией или выбросом (от англ. outlier) \cite{What-is-anomalies} понимают любой несогласованный или избыточный экземпляр выборки, отличающийся от базовой модели. Обычно это происходит, когда данные отклоняются от установленного набора данных по различным причинам, например, при неполной загрузке или неожиданном удалении информации в базе данных. Обнаружение аномалий помогает отладить процесс выявления выбросов и их устранения во избежание дефектов в наборе данных.
В условиях постоянного роста объема данных, методы машинного обучения \cite{Outliers-in-data} обеспечивают эффективную обработку больших массивов информации. Также важным преимуществом является способность выявлять сложные и неочевидные закономерности в данных, что делает их более эффективными в обнаружении аномалий.

С появлением больших данных и широкого использования информационных технологий возросла необходимость в создании эффективных \cite{AI-and-ML-for-AD} средств обнаружения аномалий в данных. В зависимости от предметной области и конкретной информационной системы, аномалии могут быть признаком несанкционированного доступа, вредоносной активности или технических сбоев.

В целом, некорректные данные могут привести к неправильным аналитическим выводам, неправильным рекомендациям и даже серьезным финансовым и организационным потерям. Так недавнее исследование SAS (американская компания-разработчик систем класса Busyness Intelligence и ПО для статистического анализа) \cite{SAS-Data-Quality} в Европе показало растущую важность качества данных в финансовой сфере. 66\% европейских компаний подтвердили, что ошибки в данных негативно влияют на прибыль, 74\% процента из них приняли меры по решению проблем качества данных. В связи с этим возникает потребность в совершенствовании методов анализа данных, способных оперативно выявлять нештатные ситуации.

В частности, обнаружение аномалий методами машинного обучения имеет широкий спектр применения и приложений: обнаружение инцидентов информационной безопасности по аномалиям \cite{Method-detection-incidents-IS}, выявление неисправностей в технических системах и IoT \cite{Diagnostics-IoT}, выявление мошеннической деятельности и фальсификаций \cite{Detection-anomalies-and-falsifications}. Выбор методов на основе ML также связан с способностью обнаружения неявных и сложных закономерностей в данных \cite{ML-powered-AD}, что обеспечивает эффективное выявление аномалий, которые могли бы быть упущены традиционными методами. Возможность обучаться на основе больших объемов данных, является эффективной в обнаружении новых, ранее неизвестных типов аномалий.

Место прохождения практики – Научно-учебный комплекс «Информатика и системы управления» МГТУ им. Н.Э. Баумана. Период прохождения практики – с 7 февраля 2025 г. по 7 марта 2025 г.

Цели прохождения практики – сформулировать и предложить варианты решения математической постановки задачи обнаружения аномалий в сетевом трафике на основе машинного обучения.

Задачи прохождения практики:

\begin{itemize}[leftmargin=0pt,itemindent=4.6em]
    \item[$\bullet$] Предложить формулировку математической постановки задачи обнаружения аномалий в сетевом трафике на основе машинного обучения.

    \item[$\bullet$] Рассмотреть, проанализировать и предложить возможные пути решения поставленной задачи.

    \item[$\bullet$] Представить выбор и обоснование решения математической постановки настоящей задачи.
\end{itemize}