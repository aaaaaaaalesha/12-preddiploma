\section{Пути решения поставленной задачи}

В данном разделе представлены некоторые методы и подходы к решению сформулированной выше задачи. 

Первый подход, который будет рассмотрен, это классический статистический метод -- критерий <<трёх сигм>>, основанный на предположении о нормальном распределении данных и позволяет выявлять аномалии, находящиеся за пределами трех стандартных отклонений от среднего значения. Несмотря на свою простоту и легкость в реализации, данный метод имеет ограничения, особенно в случаях, когда данные не подчиняются нормальному распределению.

Следующим методом является одноклассовый метод опорных векторов (One-Class SVM). Этот подход позволяет выделять области с высокой плотностью данных и эффективно обнаруживать аномалии, используя концепцию гиперплоскостей в пространстве признаков. One-Class SVM демонстрирует высокую эффективность в задачах, где аномальные данные редки и сложно выделяются.

Третий метод, который будет обсужден, это изолирующий лес (Isolation Forest). Этот алгоритм основан на принципе, что аномалии легче изолировать, так как они редки. Изолирующий лес использует случайные деревья для разделения данных и позволяет быстро и эффективно обнаруживать аномалии в больших объемах данных. В дополнение к этому, будет рассмотрено расширение данного метода — Deep Isolation Forest, которое улучшает результаты за счет использования нейронных сетей.

Также будет представлен алгоритм ECOD (Empirical-Cumulative-distribution-based Outlier Detection), который использует эмпирическую кумулятивную функцию распределения для оценки вероятности появления аномалий. Этот метод отличается простотой и отсутствием гиперпараметров, что делает его удобным для практического применения.

Наконец, в разделе будет рассмотрено применение автокодировщиков для обнаружения аномалий. Автокодировщики представляют собой нейронные сети, которые обучаются восстанавливать нормальные данные и могут быть использованы для выявления аномалий на основе ошибки восстановления. Этот подход позволяет эффективно обрабатывать сложные данные и выявлять аномалии, которые не были представлены в обучающем наборе.