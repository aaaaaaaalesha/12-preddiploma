\structure{ОСНОВНАЯ ЧАСТЬ}

В рамках прохождения преддипломной практики был подготовлен настоящий отчёт о проделанной работе. Основная часть состоит из четырёх основных разделов, затрагивающих формулировку и решение математической постановки задачи обнаружения аномалий в сетевом трафике на основе машинного обучения.

Основными объектами изучения стали актуальные подходы к детектированию аномалий сетевых пакетов, выбор оптимального алгоритма решения данной проблемы, а также математический аппарат, используемый применительно к формализации и описанию рассматриваемой задачи.

В первом разделе представлена характеристика организации Научно-учебного комплекса «Информатика и системы управления» (НУК ИУ) МГТУ им. Н.Э. Баумана, где описаны цели и задачи данного подразделения, а также его роль в образовательном и научном процессах.

Во втором разделе формулируется математическая постановка задачи, где вводится множество экземпляров сетевой активности и их признаковое описание. Определяются метки классов для аномальных и нормальных экземпляров, а также формулируется функция принятия решений, позволяющая классифицировать данные. Также рассматриваются ошибки I-го и II-го рода, их взаимосвязь, позволяющую выйти на окончательную формулировку задачи. 

Следующий, третий раздел посвящён обзору и описанию различных методов и подходов к решению сформулированной задачи. В этом разделе производится анализ существующих методов, таких как критерий <<трёх сигм>>, одноклассовый метод опорных векторов, изолирующий лес и автокодировщики. В заключение, сделан выбор в пользу одного из методов, обоснованный его эффективностью и соответствием требованиям задачи. Это позволит не только продемонстрировать практическую значимость проведённой работы, но и предложить рекомендации по дальнейшему развитию и улучшению систем обнаружения аномалий в сетевом трафике.
