\section{Характеристика организации}

Научно-учебный комплекс «Информатика и системы управления» \cite{Requlations-SEC-ISC} (далее --- НУК ИУ) является самостоятельным структурным подразделением МГТУ им. Н.Э. Баумана (далее --- Университет). Полное наименование НУК ИУ на английском языке Scientific Educational Complex «Informatics and Control Systems» Bauman Moscow State Technical University. Сокращённое --- SEC ICS BMSTU.

НУК ИУ возглавляет руководитель НУК ИУ д.т.н., профессор и заведующий кафедрой ИУ6 («Компьютерные системы и сети») Пролетарский А.В., который непосредственно подчиняется действующему ректору Университета к.т.н. Гордину М.В..

Цель деятельности НУК ИУ: обеспечение образовательного и научного процессов в соответствии с Уставом Университета. К основным задачам подразделения относят:

\begin{itemize}[leftmargin=0pt,itemindent=4.6em]
    \item Подготовку бакалавров, специалистов, магистров, научных и научно-педагогических кадров высшей квалификации в соответствии с государственными лицензиями, выданными Университету. 
        
    \item Выполнение фундаментальных, поисковых и прикладных научных исследований, проведение опытно-конструкторских работ и производство опытных образцов перспективной техники и технологий. 
        
    \item Написание и издание учебников, учебных пособий и монографий. 
        
    \item Развитие научных, педагогических и инженерных школ НУК ИУ. 

    \item Разработку и внедрение прогрессивных форм, методов и средств подготовки специалистов. 

    \item Повышение квалификации и переподготовка научно-педагогических кадров технических и экономических учебных заведений, а также специалистов предприятий с высшим образованием. 

    \item Развитие материально-технической базы НУК ИУ. 

    \item Организацию научной работы студентов. 

    \item Развитие различных форм сотрудничества с высшими учебными заведениями России и зарубежных стран. 

    \item Социальную поддержку и защиту преподавателей, научного и отвечающих требованиям подготовки вспомогательного персонала. 
\end{itemize}
