\conclusion

В данной работе была сформулирована математическая постановка задачи обнаружения аномалий в сетевом трафике на основе машинного обучения. Были предложены пути решения поставленной задачи, выбран и описан оптимальный алгоритм для её решения. В результате решение настоящей задачи будет построено на базе нейронной сети автокодировщика, хорошо зарекомендовавшего себя по результатам опубликованных работ и численных экспериментов, проводимых ранее в рамках НИРС.

Специфика предметной области и сам тип аномалий представляет особые вызовы для существующих методов их обнаружения. На основе анализа проблемы, технических и функциональных особенностей, реализуемых в информационной системе, делается выбор в пользу тех или иных методов решения поставленной задачи. Действительно, вредоносная активность имеет тенденцию выглядеть нерегулярно по сравнению с повседневными операциями, что позволяет свести данную проблему к задаче поиска аномалий в данных сетевого трафика, поступающего как из сети Интернет, так и циркулирующего в контуре организации.

Системы, основанные на заранее настроенных сигнатурах и правилах, зачастую будут <<слепы>> к атакам нулевого дня, в то время как anomaly-based-решения способны выявить подозрительную активность на основе алгоритмов машинного обучения, что будет отражено в выпускной квалификационной работе специалиста.
